\chapter*{List of Variables}

\begin{xltabular}{\textwidth}{ l  X }
	\toprule
	
	$\mathcal{A}$, $\mathcal{B}$, $\mathcal{T}$
	&
	Calligraphic letters represent polyominoes.
	Letters $\mathcal{A}$ and $\mathcal{B}$ are given to polyominoes that are about to be connected in a local plan.
	$\mathcal{T}$ indicates the target polyomino for assembly in a global plan.
	\\ \midrule
	
	$c$, $c_\mathcal{A}$
	&
	Magnetic modular cube. $c_\mathcal{A}$ would indicate that $c$ is part of the polyomino $\mathcal{A}$.
	\\ \midrule
	
	$r_C$
	&
	The cube radius is the half length of a cube face.
	All cubes in a workspace are the same size.
	\\ \midrule
	
	$r_M$
	&
	The magnet radius is the distance from the center of the cube to the center of a embedded permanent magnet.
	\\ \midrule
	
	$m_C$
	&
	Mass of a magnetic modular cube.
	\\ \midrule
	
	$p_c$, $p_\mathcal{A}$ 
	&
	Workspace position of a cube $c$ or a polyomino $\mathcal{A}$.
	In both cases position is the center of mass.
	\\ \midrule
	
	$r_{c_\mathcal{A}}$  
	&
	Vector pointing from the polyomino's center of mass $p_\mathcal{A}$ to the cube's center of mass $p_{c_\mathcal{A}}$. 
	\\ \midrule
	
	$d(c_1, c_2)$  
	&
	Euclidean distance between the centers $p_{c_1}$ and $p_{c_2}$ of the cubes $c_1$ and $c_2$.
	\\ \midrule
	
	$\vec{N}$, $\vec{E}$, $\vec{S}$, $\vec{W}$
	&
	Cardinal direction vectors dependent on the longitude orientation of the global magnetic field.
	\\ \midrule
	
	$e$, $e_\mathcal{A}$
	&
	Side face of a magnetic modular cube represented by a vector.
	$e \in \{ \vec{N},\vec{E},\vec{S},\vec{W}\}$ due to the assumption of cubes being always aligned with the magnetic field. 
	$\lVert e \rVert = 1$ holds true and $e_\mathcal{A}$ indicates that $e$ belongs to a cube contained in polyomino $\mathcal{A}$.
	\\ \midrule
	
	$n$
	&
	Size of the target polyomino or number of cubes in the workspace.
	In case of our global planner cube count equals target polyomino size.
	\\ \midrule
	
	$\vec{d}$
	&
	Displacement vector for one pivot walking cycle of a polyomino.
	\\ \midrule
	
	$\vec{a}$
	&
	Pivot walking axis of a polyomino in global coordinate frame.
	Vector between north and south pivot point.
	\\ \midrule
	
	$\alpha$
	&
	Pivot walking angle.
	\\ \midrule
	
	$\vec{w}$
	&
	Pivot walking direction $\vec{w} \in \{ \vec{E}, \vec{W} \}$.
	\\ \midrule
	
	$\vec{m}$
	&
	Slide-in direction $\vec{m} \in \{ \vec{E}, \vec{W} \}$.
	\\ \midrule
	
	$\overrightarrow{\mathcal{A}\mathcal{B}}$
	&
	Vector used in the process of aligning cubes.
	Points from $p_{c_\mathcal{A}}$ to $p_{c_\mathcal{B}}$ for straight aligning, or to a position above/below $p_{c_\mathcal{B}}$ for offset aligning.
	\\ \midrule
	
	$d_\textit{offset}$
	&
	Offset distance for offset aligning.
	$d_\textit{offset} > 2 r_C$.
	\\ \midrule
	
	$\beta$  
	&
	The rotation angle is a change in longitude orientation of the global magnetic field.
	\\ \midrule
	
	$\mathbf{R}_\beta$  
	&
	$2 \times 2$ rotation matrix for rotating vectors by an angle of $\beta$.
	\\ \midrule
	
	$\#\textit{steps}$
	&
	Number of estimated pivot walking cycles in our dynamic walk-align-realign approach.
	\\ \midrule
	
	$s$
	&
	Plan state of either local or global plans. States if successful, or the reason of failure.
	\\ \midrule
	
	$A$
	&
	Sequence of actions $a_1, ... , a_k$ a local plan consists of.
	\\ \midrule
		
	$g$, $g_\textit{init}$, $g_\textit{goal}$
	&
	Configurations of the configuration-space $\textit{SE}(2)$. $g_\textit{init}$ indicates the initial and $g_\textit{goal}$ the goal configuration of local or global plan.
	\\ \midrule
	
	$S$, $S(g)$, $S_\mathcal{T}$, $S_\textit{trivial}$
	&
	Polyomino sets store information about the polyomino types present in the workspace without considering position or distinguishing between physical polyominoes.
	The amount of one type is also stored.
	$S(g)$ is the polyomino set of a configuration $g$.
	$S_\mathcal{T}$ contains only one occurrence of $\mathcal{T}$ and $S_\textit{trivial}$ only trivial polyominoes.
	Both are used in TCSA graphs.
	\\ \midrule
	
	$\hat{n}$
	&
	Maximum polyomino size in one configuration or polyomino set.
	\\ \midrule
	
	$t_c$
	&
	Continuous two-cutting edge path through a polyomino.
	\\ \midrule
	
	$G_{\textit{TCSA}}(\mathcal{T})$
	&
	Two-cut-sub-assembly graph of $\mathcal{T}$ represented by nodes $V$ and edges $E$.
	Nodes are polyomino sets and edges connect two sets $\{S_0, t_c, S_1\}$ with a two-cut as an edge weight.
	\\ \midrule
	
	$L_\mathcal{A}$
	&
	Collection of all physically distinct polyominoes of the polyomino type $\mathcal{A}$.
	\\ \midrule
	
	$O$
	&
	List of connection options $o$ for one configuration determined with a TCSA graph.
	\\ \midrule
	
	$\hat{o}(o_1,o_2)$
	&
	Function comparing connection options $o_1$ and $o_2$ and returning the better one based on the option sorting strategy used.
	\\ \midrule
	
	$P$
	&
	Plan stack containing a continuous sequence of local plans $p$.
	Used in the global planning algorithm.
	\\ \midrule
	
	$\#\textit{local}$
	&
	Number of local plans simulated during planning with the global planning algorithm.
	\\ \midrule
	
	$\#\textit{config}$
	&
	Number of configurations explored during planning with the global planning algorithm.
	\\ \midrule
	
	$\mu_\textit{mag}$
	&
	Magnetic strength of embedded permanent magnets of magnetic modular cubes use in our simulator.
	\\ \midrule
	
	$\mu_\textit{field}$
	&
	Strength of the magnetic field used in our simulator.
	\\ \midrule
	
	$p_\textit{fric}$
	&
	Friction point of a cube depending on the latitude of the magnetic field.
	Either at the position of north or south magnet.
	\\ \midrule
	
	$n_\textit{fric}$
	&
	Number of friction-cubes of a polyomino to which friction force is applied.
	\\ \midrule
	
	$w_\textit{nom}$
	&
	Fraction of nominal friction that gets applied to all cubes of a polyomino.
	\\ \bottomrule
	
\end{xltabular}
\chapter*{Abstract}

In this thesis we developed a heuristic approach for the motion planning problem of assembling structures with magnetic modular cubes, developed and researched by Bhattacharjee et al.\ \cite{Bhattacharjee2022}, in the 2-dimensional special Euclidean group, the space of rigid movements in a 2-dimensional plane.
Magnetic modular cubes are cube-shaped bodies with embedded permanent magnets uniformly controlled by a global time-varying magnetic field surrounding the workspace.

A 2D physics simulator is used to simulate global control and the resulting continuous movement of magnetic modular cube structures as well as magnetic attraction and repulsion, while detecting and resolving collisions.
The simulator allows closed-loop control algorithms for planning the connection of two structures at desired faces.
These developed sequences of movements, called \textit{local plans}, will be used on a global scale to plan the assembly of specified target structures in a rectangular workspace with no internal obstacles.
The assembly is done by generating a building instructions graph for a target structure that we traverse in a depth-first-search approach by applying local plans to current states of the workspace.

We analyze how target structures of varying sizes and shapes in different rectangular workspaces affect planning time and the rotational-cost of movements.
The traversal of the building instruction graph can be further optimized, for which we present three strategies and their effect on the performance of the global planner.
The majority of randomly created instances in our experiments can be solved in under $200$ seconds for structures of up to $12$ cubes, but certain attributes of target structures can drastically decrease the efficiency of the global planner.
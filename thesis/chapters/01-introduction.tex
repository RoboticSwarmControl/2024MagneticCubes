\chapter{Introduction}
\label{chap:intro}

Self-assembling modular parts forming bigger structures is a well-known concept in nature.
Most functionalities of living organisms follow this principle \cite{bishop2005}.
DNA, for example, has the ability to self-replicate by using differently shaped proteins that combine themselves in various ways.
At larger sizes, these cells can be combined to assemble tissue, organs and even whole organisms.
Complex structures, like proteins, can be assembled and disassembled depending on the task they should accomplish at a given point in time. 
Using self-reconfiguring robot swarms in such a way has promising applications in the future.
Biomedical applications could be targeted drug delivery or drug screening \cite{sitti2015}, or a robot swarm could be used for milliscale and microscale manufacturing \cite{pelrine2016}.

Designing robots at these small sizes faces challenging problems.
Equipping each robot with its own sensors, actuation-system, connection-system and power supply seem infeasible, in terms of the miniaturization required and power-limitations \cite{white2007}.

Therefore, the use of external global control, effecting every robot in the workspace with the same torque and force, is a promising solution \cite{white2007}.
Using robots with embedded permanent magnets, has all the desired effects.
Robots can be controlled by an external magnetic field and also connect to each other without any internal power supply and for sensing an external camera can be used \cite{saab2019}.

One example for magnetically controlled robots are the magnetic modular cubes by Bhattacharjee et al.\ \cite{Bhattacharjee2022}, which are the subjects of this thesis.
We will look at the difficulties and problems that occur, when assembling structures with magnetic modular cubes in the 2-dimensional special Euclidean group $\textit{SE}(2)$, the space of rigid movements in a 2-dimensional plane.

\section{Related Work}

Continuous motion planning is a crucial subject in the field of robotics.
The goal is to find a path from the initial state of a robot to a desired goal state, by performing actions which the robot is capable of.
The movement may result in collision with static obstacles and with other robots, but the objects may not overlap.
The state of the system is also called a configuration.
All possible configurations one or multiple robots can be in is defined as the configuration-space.
Motion planning complexity is often exponential in the dimension of the configuration space \cite{LaValle2006}.
Increasing the number of robots and/or possible actions, increases the dimension of the configuration space.
It is difficult to engineer algorithms that explore these huge configuration-spaces and provide continuous path from the initial to the goal configuration, or report failure, if the goal is not reachable.
Decades of research has been done on motion planning.
The textbooks \cite{LaValle2006} and \cite{Mueller2019} offer a great overview and also explain a lot of important concepts in detail.

When working with configuration-spaces that are uncountable infinite, like the special Euclidean group, one concept that has been successful for many robotics problem is sample-based motion planning.

By taking samples, you can reduce the planning problem from navigating a configuration space to planning on a graph, but you might lose possible solutions.
Algorithms like that are not complete anymore, but by using a good sampling technique you can get arbitrarily close to any point.
Ways of sampling include random sampling being probabilistically complete or using a grid with a resolution that is dynamically adjustable resulting in resolution completeness.
After sampling, conventional discrete planning algorithms can be applied \cite{LaValle2006}.

One state-of-the-art sampling-based approach uses rapidly-exploring random trees (RRT).
This method tries to grow a tree-shaped graph in the configuration space by moving into the direction of randomly chosen samples from already explored configurations. That way the space gets explored uniformly without being too fixated on the goal configuration \cite{lavalle1998,lavalle2001}.

When working with multiple robots, the interaction of robots with each other becomes important.
One interesting idea is that single robots can connect to form bigger structures.
This is referred to as self-assembly and E. Winfree \cite{winfree1998} proposed the abstract Tile Assembly Model (aTAM) in the context of assembling DNA.
In this model, particles can have different sets of glues and connect according to certain rules regarding the glue type.
However, he considers this process as nondeterministic, so there is no exact instruction on how to assemble a desired structure.

One model more related to the magnetic modular cubes is the Tilt model from Becker et al.\ \cite{Becker2014_SP}.
In the Tilt model, all tiles move into one of the cardinal directions until hitting an obstacle.
Different variations of the model include moving everything only one step, or the maximally possible amount.
It offers a solution for motion planning problems when robots are controlled uniformly by external global control inputs.

In \cite{Becker2014_SP} it is shown that transforming one configuration into another, known as the reconfiguration-problem, is NP-hard.
Caballero et al.\ \cite{caballero2020} also researched complexity of problems regarding the Tilt model.
Following work \cite{Becker2014} also proves that finding an optimal control sequence, minimizing the number of actions, for the configuration-problem is PSPACE-complete.
Furthermore, research is done on designing environments in which the Tilt model can be used to accomplish certain tasks.
In particular, Becker et al.\ \cite{Becker2014} create connected logic gates that can evaluate logical expressions.

More on the side of self-assembly, in \cite{Becker2020} the construction of desired shapes using the tilt model is researched.
It presents a method that can determine a building sequence for a polyomino by adding one tile at a time, considering the rules of Tilt.
Also examined are ways of modifying the environment to create factories that construct shapes in a pipeline by repeating the same global control inputs.
Shapes can be constructed more efficiently by combining multi-tiled shapes to an even bigger structure.
One article considering the construction with so-called sub-assemblies is proposed by A. Schmidt \cite{Schmidt2018}.

Most recently, Bhattacharjee et al.\ \cite{Bhattacharjee2022} developed the magnetic modular cubes.
These robots contain embedded permanent magnets and have no computation or power supply.
Instead, they are controlled by an external time-varying magnetic field and are able to perform various actions.
Most importantly, they can rotate in place or use a technique called pivot walking to move either left or right.
The magnets also act as glues and allow the cubes to perform self-assembly.
Although it is theoretically possible to assemble 3-dimensional structures, most research was done by only connecting cubes in two dimensions.
Since all cubes are the same size, the assembled 2-dimensional shapes can be represented as polyominoes.
An enumeration was done on the amount of possible polyominoes that can be created by cubes with different magnet configurations \cite{Lu2021}.

By limiting the controls to only 90 degree turns and assuming a uniform pivot walking distance for all structures per step, magnetic modular cubes follow rules similar to the Tilt model.
Following these limitations, a simple discrete open-loop motion planer was developed, that explores a finite configuration-space and lists all the possible polyominoes that can be created from an initial configuration \cite{Bhattacharjee2022}.

One interesting paper from Blumenberg et al.\ \cite{blumenberg2023} explores the assembly of specific target polyominoes in arbitrary environments, when cubes obey the tilt model in a discrete setting.
He provides different algorithmic approaches using various distance heuristics and a solution making use of RRTs.

Lu et al.\ \cite{Lu2023} are working on establishing closed-loop control for magnetic modular cubes by using computer vision-based feedback of the workspace.
The used motion planner is still working in a discrete setting, but can assembly specific target shapes, while handling collision events of estimated continuous motion.


\section{Contribution}

We provide more information about the general framework of magnetic modular cubes on which this thesis relies in \autoref{chap:prelim}.

In \autoref{chap:local} we develop a local planner that is connecting two magnetic modular cube structures at desired faces.
We do this with a closed-loop algorithm to account for all types of collision events.
The resulting local plans are not optimal, but follow heuristics for minimizing the cost of movements.
The local planner focuses more on reducing planning time. 

The local planner works with our magnetic modular cube simulator presented in \autoref{chap:sim}.
This 2D-physics simulator aims to replicate the behavior of magnetic modular cubes accurately, while still being efficient enough to be used for motion planning.
The simulator does not assume discrete movement or limits rotations to a certain amount.

Based on the local planner we develop a global planning algorithm in \autoref{chap:global}, which provides a control sequence to assemble desired target structures.
The configuration-space is sampled by only considering connections between structures as local plans and using a specially constructed graph as a building instruction for target polyominoes.
The use of RRTs would be too inefficient, since we are working with a high fidelity simulation. 

Results evaluating our global planner are presented in \autoref{chap:results} and a conclusion with possible directions for future work is given in \autoref{chap:conclusion}.
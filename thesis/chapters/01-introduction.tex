\chapter{Introduction}

Motivation, applications...

\section{Related Work}

Motion planning is a crucial subject in the field of robotics. The goal is to change the initial state of a robot to a desired goal state, by performing actions which the robot is capable of. The state of the system is also called a configuration and all possible configurations a robot can be in is defined as the configuration-space.
The dimension of the configuration-space gains rapidly in complexity by increasing the number of robots and static obstacles. It becomes necessary to avoid collision between those.
It is difficult to engineer algorithms that explore these huge configuration-spaces and provide a sequence of actions to perform to reach the goal configuration or report failure, if the configuration is not reachable. A lot of research was done on motion planning for a great overview and detailed description of concepts you can see \cite{LaValle2006} and \cite{Mueller2019}.

\begin{comment}
Different methods for motion planning:

simple discrete planning

sample-based planning

combinatorial planning


Tilt assembly 

- global control ...
- Applications	
- reconfiguration problem PSPACE Complete

Construction of shapes

- what can be done with complex enviroment
- on tile at a time
- sub-assemblies

Magnetic cubes

- explain cubes with magnets, pivotwalking
- discrete planner
- enumeration of polyominoes, consider fixed polys
\end{comment}



\section{Contribution}

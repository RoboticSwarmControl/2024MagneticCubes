\chapter{Conclusion and Future Work}
In this thesis, we designed motion planning algorithms for the assembly of shapes in the Tilt model based on multiple different approaches, including search-based and sampling-based motion planning algorithms. We evaluated the effectiveness of these approaches experimentally on procedurally created boards and investigated the parameters that are most significant for the performance of motion planning algorithms in general, as well as the specific strength and weaknesses of the different approaches.
We found that the number of tiles on the board has the most significant impact on the runtime, which is an expected result since the difficulty of motion planning problems is known to largely depend on the dimensionality of the configuration space, which is proportional to the number of tiles in the case of Tilt.\par
Our first approach, an A* search in combination with a consistent heuristic based on the distance of tiles to the target shape, is able to find optimal solutions for small numbers of tiles, but quickly becomes computationally infeasible.\par
Greedy best-first search with heuristics based on the distances of tiles to the target shape is more efficient, however it often produces a solution with far from optimal length. On a positive note, the average factor by which a solution found by GAD was longer than a solution to the same instance found by AD was less than 3 for the evaluated instances and seems to decrease for larger boards. Unfortunately, the runtime of greedy best-first search increases sharply when the number of tiles is greater than 10 and is very sensitive to additional tiles that are not needed to construct the target shape, due to the heuristics indifference towards the glue types of tiles and their possible positions within the target shape.
The three proposed greedy best-first search algorithms GGD, GAD, and WSD (with the exponent 2), which all use a heuristic based on the distance of tiles to the target shape, exhibit very similar performance. However, GGD seems to perform worse than the other algorithms as the number of tiles grows and WSD produced solutions of slightly shorter length. Overall, WSD seems to present the best tradeoff between solution length and runtime, which might be further improved by selecting a different exponent.\par
Avoiding subassemblies and adding one tile at a time to the target polyomino can often quickly yield a solution even on boards with more than 10 tiles. However, this approach is not a complete algorithm. Furthermore, it struggles with boards that are densely populated with tiles as well as boards containing few obstacles that separate tiles from each other, because in these cases it becomes hard or even impossible to avoid subassemblies. Another problem for one-tile-at-a-time motion planning algorithms is the necessity to compute a valid construction order for the target shape with regard to the glues on the edges of the available tiles. For complicated glues and large target shapes, this is a hard problem to solve. While a large number of glue types increases the difficulty of assembling the target shape, it helps to avoid the accidental creation of undesired subassemblies and can therefore be advantageous for one-tile-at-a-time algorithms, once they found a valid building order. Whereas for the other evaluated classes of algorithms, more glue types had a negative effect on the performance. \par
A simple solution shortening technique can help to decrease the length of solutions found by non-optimal algorithms and is particularly effective for solutions found with the RRT algorithm. \par
Even though the Fixed Seed Tile Polyomino Assembly Problem is PSPACE-complete, it was much easier to solve at the same numbers of tiles than its counterpart without fixed seed tiles. DFP is a simple and fast motion planning algorithm for this problem, however, it is not complete. In the experiments, DFP solved a large fraction of instances including instances with larger numbers of tiles in a short time. \par
The evaluation of computational complexity for the Polyomino Assembly Problem without a fixed seed tile and without extra tiles is left to future research. \par
RRT is a promising method to solve Tilt motion planning problems. Combined with a computationally more expensive expansion step, it has the added benefit of requiring less memory than other approaches. A major challenge in this context is to find a cost-to-go function that is fast to compute and gives a good approximation of the actual distance between two configurations. Furthermore, the RRT algorithm we designed has a large number of adjustable parameters. Future work could find optimal settings for these parameters and evaluate the effectiveness of other cost-to-go functions. \par
Better best-first search algorithms could potentially be achieved with heuristics that not only depend on the distance of tiles to the target shape but instead consider, for example, the involved glue types and possible positions of tiles within the target polyomino. \par
Another possible direction for future work is to evaluate motion planning algorithms on configurations with parameters outside the scope of this thesis, such as boards that contain a high density of tiles or a small number of obstacles.
%!TeX root=../thesis.tex
\chapter{Introduction}

Motivated by the difficulty of controlling large robot swarms at the molecular scale, where the manipulation of individual robots becomes infeasible due to their small size and limited capabilities, Becker et al. \cite{Becker2013} proposed the Tilt model in 2013. It provides a way to describe particles that move under the influence of uniform external forces, such as magnetic or gravitational fields. When combined with particles that have the ability to self-assemble into larger objects, this gives rise to several potential applications including medical applications, such as minimally invasive surgery or targeted drug delivery, as well as the construction of objects at tiny scales. \par
In the Tilt model, movable unit squares called tiles are placed on a 2-dimensional grid that may also contain immovable obstacles. All tiles receive the same global movement instructions which cause them to move a unit distance (or maximally) in one of the cardinal directions unless they are blocked by an obstacle. Additionally, tiles can have glues on their edges, which allow them to bond according to certain rules and assemble into polyominoes. Tiles forming a polyomino move as a unit and cannot be separated. \par
A difficult challenge in this context is the design of efficient motion planning algorithms that can compute a sequence of global movement instructions such that the resulting configuration contains the desired target shape. Ideally, such an algorithm should minimize the length of the computed sequences.
In this thesis, we design several heuristic methods that attempt to solve this problem and evaluate them experimentally on randomly-generated boards and initial configurations. Furthermore, we analyze the impact that several characteristics of the environment have on the efficiency of motion planning algorithms.

\section{Related Work}

Motion planning is a fundamental research topic in the field of robotics, which has been studied extensively. See \cite{LaValle2006} for a textbook or \cite{Yang2019} for a survey article. The general problem is to find a continuous path of one or multiple robots from an initial position to a target position that avoids collisions or to determine that no such path exists. Formally, this is known as a reconfiguration problem, in which a path from an initial configuration to a target configuration through a configuration space must be discovered. In this context, the configuration space is defined as the set of possible states (or configurations) of the robots. The dimensionality of the configuration space, which depends on the number of independently moving parts of the robot, often limits the efficiency of motion planning algorithms, as the number of vertices that must be explored grows exponentially in the number of dimensions. This problem is known as the \emph{curse of dimensionality}~\cite{Petrovic2018}. For example, in this thesis $n$ movable tiles corresponds to a $2n$-dimensional configuration space.
Furthermore, several motion planning problems are known to be computationally difficult. For example, the Warehouseman's Problem, in which multiple, independently moving, rectangular objects in 2-dimensional space must be moved to target locations, has been proved to be PSPACE-hard \cite{Hopcroft1984}. Nevertheless, several approaches to efficient motion planning have been developed. \par
For low-dimensional systems, in particular, \emph{search-based} planning algorithms such as A* and other heuristic tree search methods can be effective \cite{Bonet2001}. These algorithms build a graph structure over the configuration space and then use graph-search algorithms to find a path. \par
Another common class of motion planning algorithms is the \emph{potential field method}, in which an artificial potential field represents attractive and repulsive forces that guide the path of the robots. Although this method is simple and adaptable, local minima in the potential field pose a major challenge \cite{Yun1997}. \par
Lastly, \emph{sampling-based} methods like probabilistic roadmaps (PRM)~\cite{Kavraki1996} and rapidly exploring random trees (RRT)~\cite{Lavalle1998} can be used to discover a path in high-dimensional configuration spaces (e.g. humanoid robots or multi-robot systems). Sampling-based methods sample configurations at random and rely on collision checking to connect the sampled points to a graph. RRT* and related solutions try to combine the efficient exploration of sampling-based methods with the optimality of the results by continuously improving a solution. \par
This thesis uses search-based algorithms as well as RRTs to solve motion planning problems in the context of Tilt.\par

The Tilt model of motion planning~\cite{Becker2013} presents a possible solution for scenarios in which a collection of particles or robots which cannot be moved individually needs to be reconfigured. The basic idea is to control tiles on a 2-dimensional grid with fixed obstacles through the use of uniform external forces in one of the cardinal directions at a time. In practice, gravity or magnetic fields can be used for this purpose. Depending on the specific model, every control input causes all tiles to move either maximally until they hit an obstacle (tilt transformation) or by one unit distance in the corresponding direction (step transformation). \par
Later work by Becker et al. \cite{Becker2017} introduced the ability to assemble polyominoes out of tiles similar to traditional models for self-assembly, such as the \emph{abstract Tile
Assembly Model} (aTAM) introduced by Erik Winfree in 1998 \cite{Winfree1998}.
In self-assembly models like aTAM, particles combine through a nondeterministic process, and the possible products are determined only by the rules according to which particles bond. Practical implementations of self-assembling systems can use molecular diffusion of DNA or other substances to carry out computations.
aTAM uses Wang Tiles \cite{Wang1961} as atomic building blocks to form polyominoes according to the glues on each of their edges and the affinities defined by a glue function. In the original model, a seed tile is required to form assemblies and tiles must be added one at a time. Over the following years, many generalizations of aTAM were proposed \cite{Chen2011a, Cannon2012, Demaine2008}, some of which allow subassemblies to be combined in parallel or completely abandon the concept of seed tiles. \par As the above models, we use Wang tiles as building blocks and let glues on each of the edges of a Wang tile decide how tiles bond. Furthermore, both a model including a fixed seed tile that is required to form assemblies, as well as a model in which tiles can combine freely, are considered in this thesis. \par

Over the last years, the computational complexity and algorithmic solutions of multiple problems that arise in the Tilt model have been investigated. \par
Shortly after the introduction of the model, in 2014, it was shown that finding a minimum-length sequence of tilts that reconfigures an initial configuration into a target configuration is PSPACE-complete~\cite{Becker2014}. \par
In addition to the reconfiguration problem, two other naturally arising problems received attention, namely the \emph{relocation} and \emph{occupancy} problem. The first one asks whether a specific tile can reach a specific target position; the second one, whether a given position on the board can be occupied by any tile.
In 2019 Balanza-Martinez et al. \cite{Balanza-Martinez2019} showed that the relocation problem for single tiles under tilt transformations is PSPACE-complete if a single $2 \times 2$ polyomino is allowed. \par
Later, in \cite{BalanzaMartinez2020} the authors showed that the occupancy problem is PSPACE-complete, even for $1 \times 1$ tiles without glues. \par
Several papers \cite{Becker2017, Schmidt2018, Balanza-Martinez2019, BalanzaMartinez2020} dealt with the constructability of shapes under tilt transformations, as well as universal constructors, that can build arbitrary shapes. \par
Directly relevant to our work are the hardness results under single-step transformation.
In \cite{Balanza-Martinez2019a, Caballero2020b}, motion planning problems under step transformation with a limited number of allowed input directions were investigated. The relocation problem was found to be NP-complete when limited to 3 directions, whereas reconfiguration was shown to be NP-complete when limited to 2 directions. On the other hand, the occupancy problem with only $1 \times 1$ tiles is solvable in polynomial time regardless of the number of allowed directions. \par
For the case without limited directions, Caballero et al. \cite{Caballero2020a} proved that occupancy is PSPACE-complete when $1 \times 2$ polyominoes are allowed; whereas relocation is PSPACE-complete even with only $1 \times 1$ tiles. \par
Recent work on the constructibility of polyominoes under step transformations found efficient algorithms for certain classes of polyominoes but leaves open the question about the computational complexity of the general problem \cite{Keller2021}. \par
In 2020, Becker et al. investigated algorithmic solutions for collecting particles in a 2-dimensional maze through the use of uniform external forces \cite{Becker2020a}.
To the best of our knowledge, no other work on the design and experimental evaluation of Tilt-related motion planning algorithms has been published.

\section{Contribution and Structure}
Chapter 2 provides definitions and introduces two motion planning problems related to the assembly of polyominoes in the Tilt model under step transformations: the Polyomino Assembly Problem and the Fixed Seed Tile Polyomino Assembly Problem.
Both are special cases of a reconfiguration problem that differ in their assumptions about how tiles bond. The first one allows polyominoes to bond whenever the glues on their edges allow it, the second one requires the presence of an immovable seed tile to assemble polyominoes. \par
In Chapter 3, we discuss the hardness of the problems and prove PSPACE-hardness for the Fixed Seed Tile Polyomino Assembly Problem, even under the limitations that the number of tiles on the board is limited to the size of the target shape and the number of glues is constant.
Subsequently, several algorithmic approaches based on heuristics that aim to solve these problems in practice are introduced. Both search-based methods, such as A* search and a sampling-based method using RRTs are explored. Furthermore, a method to shorten an existing step sequence is briefly discussed. \par
In Chapter 4, we evaluate the effectiveness of the proposed algorithmic approaches on randomly generated instances and investigate the effect of different parameters on the performance of each algorithm and the practical difficulty of the problems in general.\par
Chapter 5 concludes the thesis and gives possible directions for future research.
\chapter*{Abstract}
In this thesis, we investigate motion planning algorithms for the assembly of shapes in the Tilt model in which particles move under the influence of uniform external forces and self-assemble according to certain rules. The goal is to design algorithms that can efficiently compute short input sequences that transform an initial configuration of movable square-shaped tiles on a 2D grid with immovable obstacles into another configuration containing a target shape. Furthermore, the influence of various input parameters on the performance of motion planning algorithms is evaluated.\par
\begin{sloppypar}
We discuss the computational complexity of the underlying problems and prove PSPACE-completeness when the assembly of shapes requires the presence of a fixed seed tile, even under the limitation that the number of available tiles is equal to the size of the target shape.
Based on methods for motion planning from the field of robotics, we design various algorithms. Approaches based on an A* search with a consistent heuristic can compute a solution of minimal length, whereas other approaches, such as greedy best-first search and rapidly-exploring random trees, give up optimality in exchange for improved runtime efficiency. Additionally, a method that can shorten an existing input sequence is proposed.
The performance of the proposed algorithms is experimentally evaluated on sets of instances created randomly based on six input parameters. We analyze the influence of the parameters on the performance of motion planning algorithms and demonstrate the strength and weaknesses of each approach.\par
 \end{sloppypar}
For all evaluated algorithms, the runtime largely depends on the number of tiles. Algorithms that assemble the target shape one tile at a time displayed the overall best performance, even though they are not a complete solution. A problem variant that requires a fixed seed tile is easier to solve in practice and can be solved more efficiently by a specialized algorithm.